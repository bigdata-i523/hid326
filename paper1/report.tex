\documentclass[sigconf]{acmart}

\input{format/i523}

\begin{document}
\title{Big Data in Clinical Trials}
\author{Mohan Mahendrakar}
\affiliation{%
  \institution{Indiana University}
  \streetaddress{P.O. Box 1212}
  \city{Bloomington} 
  \state{Indiana} 
  \postcode{43017-6221}
}
\email{mmahendr@iu.edu}

% The default list of authors is too long for headers}
\renewcommand{\shortauthors}{B. Trovato et al.}


\begin{abstract}
Will understand about Clinical Trials and how Big
data is impacting Clinical Trials. Clinical Trials is experiencing a
data-driven transformation. Clinical trials getting ready with new and 
ever extra efficient molecular and computer technologies, we are entering
new era where big data technologies are helping us forefront the contest 
against various deceases. This technology driven information bang, often 
denoted to as ``big data''.

\end{abstract}

\keywords{i523, HID326, Big data, Clinical, Trials, Health care, Data
integration, Analytics}

\maketitle

\section{Introduction}
A prime focus of clinical trials is gaining knowledge from studying
a group of patients which can then be functional to a much wider group of
patients to recover from deceases. In general practice, providing care 
to patients is delivered within a rich background of intrinsic and
endemic confusing issues and prejudices related with practices and 
patients\cite{TR02}.

The data received from around the world from various patients, 
decease form Big data, Big data nothing but collection of large 
data sets. These data sets may grow even beyond petabytes in size.  
In clinical trials big data usage just started and but big data use
cases are promising and widely used in near future. Few hail the great
use cases and some are neutral about big data but big data is game 
changing technology in clinical trials\cite{TR05}. 

%%%%%%%%%%%%%%%%%%%%%%%%%%%%%%%%%%%%%%%%%%%%%%%%%%%%%%%%%%%%%%%%%%%
According to \cite{TR07} digital world grows at aspect of 300, to 
40,000 exabytes from 130 exabytes. until year 2020  digital world 
keep doubled every 2 years. 

According to \cite{TR03} it is predicted that the market for Big Data
technology and services will reach \$20 billion in 2017, up from 
\$3.2 billion in 2010. This is an annual growth rate of 40 percent,
which is about seven times the rate of the overall information and
communications technology market. According to CB insights, health
care investments in Big Data totaled \$274.5 million in 2012, and 
it went to \$371.5 million in 2013.

%%%%%%%%%%%%%%%%%%%%%%%%%%%%%%%%%%%%%%%%%%%%%%%%%%%%%%%%%%%%%%%%%%%
\section{Big Data \& clinical research}
Determining clinical trials hidden data patterns and relations within
the mixed data, discovering new pharma companies and drug goals.
Letting the new development of predictive disease progression models.
Analyzing Real World Data (RWD) as a balancing instrument to 
clinical trials, for the rapid development of new personalized 
medicines. The expansion of progressive statistical methods for 
learning fundamental relations from large scale observational data is a
very important factor in the analysis\cite{TR04}.

\subsection{Data Integration}
%%%%%%%%%%%%%%%%%%%%%%%%%%%%%%%%%%%%%%%%%%%%%%%%%%%%%%%%%%%%%%%%%%%
Having access to right, appropriate and trustable and associated is a 
biggest challenges facing medical clinical trials. The
ability to accomplish and integrate data collected at all phases of the
clinical trials right from detection to real world usage after regulatory 
approval, this is a essential goal to let organizations to originate
more profit from big data technologies. Value addition analytics are designed
on data as the foundation.  The trusted sources of all pieces are formed 
based on effective end-to-end data integration and relates dissimilar data
irrespective of source that can be internal or external, publicly available or patented.
Another benefit of data integration is one can perform wide-ranging 
searches for subsections of information based on the relations established 
rather than only available data.  ``Smart'' algorithms which connects 
to clinical trials information and laboratory could generate automatic
reports that helps to find right applications or compounds and even 
generate flags which helps in understanding safety or effectiveness
\cite{TR02}.
%%%%%%%%%%%%%%%%%%%%%%%%%%%%%%%%%%%%%%%%%%%%%%%%%%%%%%%%%%%%%%%%%%%
Applying data integration end-to-end needs a lot of competences,
including but not limited certified sources of documents
and data, the capability to create cross relations among the elements, 
robust quality assurance, workflow management, and accesses based 
on roles to safeguard that definite data elements are available only 
to authorized to access and see it. Medical organizations usually evade
overhauling their entire data-integration systems at one point of time
due to the logistical challenges and costs associated, even though
there are few  international pharmaceutical enterprises employing 
a ``big bang'' method to redesigning its clinical IT systems\cite{TR02}.


%%%%%%%%%%%%%%%%%%%%%%%%%%%%%%%%%%%%%%%%%%%%%%%%%%%%%%%%%%%%%%%%%%%
Data is being generated by different sources and comes in a variety
of formats including unstructured data. All of this data needs to 
be integrated or ingested into big Data Repositories or Data 
Warehouses. This involves at least three steps, namely, Extract, 
Transform and Load (ETL). With the ETL processes that have to be 
tailored for medical data have to identify and overcome structural,
syntactic, and semantic heterogeneity across the different data 
sources. The syntactic heterogeneity appears in forms of different 
data access interfaces, which were mentioned above, and need to be
wrapped and mediated. Structural heterogeneity refers to different
data models and different data schema models that require 
integration on schema level. Finally, the process of integration 
can result in duplication of data that requires consolidation\cite{TR04}.

%%%%%%%%%%%%%%%%%%%%%%%%%%%%%%%%%%%%%%%%%%%%%%%%%%%%%%%%%%%%%%%%%%%
The process of data integration can be further enhanced with 
information extraction, machine learning, and semantic web 
technologies that enable context based information interpretation.
Information extraction will be a mean to obtain data from additional
sources for enrichment, which improves the accuracy of data 
integration routines, such as duplication and data alignment. 
Applying an active learning approach ensures that the deployment of
automatic data integration routines will meet a required level of 
data quality. Finally, the semantic web technology can be used to 
generate graph based knowledge bases and oncologies to represent 
important concepts and mappings in the data. The use of standardized 
oncologies will facilitate collaboration, sharing, modelling, and 
reuse across applications\cite{TR04}.

%%%%%%%%%%%%%%%%%%%%%%%%%%%%%%%%%%%%%%%%%%%%%%%%%%%%%%%%%%%%%%%%%%%
\subsection{Exascale computing}
After data integration is completed, the big question is how to process
such huge volume of the data? There will be use cases, e.g. precision
medicine, where the promises brought by big data will only be fulfilled
through dramatic improvements in computational performance and capacity,
along with advances in software, tools, and algorithms. Exascale
computers-machines that perform one billion calculations per second and
are over 100 times more powerful than today's fastest systems will be
needed to analyses vast stores of clinical and genomic data and develop
predictive treatments based on advanced 3D multi-scale simulations with
uncertainty quantification. Precision medicine will also require scaling
these systems down, so clinicians can incorporate research breakthroughs
into everyday practice\cite{TR04}.
%%%%%%%%%%%%%%%%%%%%%%%%%%%%%%%%%%%%%%%%%%%%%%%%%%%%%%%%%%%%%%%%%%%

\subsection{Data-driven metamorphosis}
Data collected in clinical trials experiencing a data driven
metamorphosis. Information technologies equipped with new
and even more efficient molecular, we are in the era where information
is supporting us driving force to fight against to deceases like cancer. 
This expertise driven data blast, generally mentioned as ``big data'', 
is not only helping discoveries in biomedical, then it is also rapidly 
applying the practice of oncology into an information science. This 
development is very critical, as outcomes to-date have opened the enormous
complication and genetic heterogeneity of trials patients and patients 
tumors, a soberingnotice of the challenge undergoing each patient and
their oncologist. The answer to this issue is addressed only through 
developing data analytics in clinico-molecular, that will help deeper
analyses of mechanics which is controlling the biological and clinical
response to available therapeutic options. Beyond the available 
guidelines for better-quality patient care, such progressions in 
predictive and evidence-based analytic stand to deeply impact the 
existing processes in discovering the drugs for cancer drug and also
corresponding clinical trials \cite{TR01}.

%%%%%%%%%%%%%%%%%%%%%%%%%%%%%%%%%%%%%%%%%%%%%%%%%%%%%%%%%%%%%%%%%%%
\subsection{Big data analytics}
Medical research has always been a data-driven science, with 
randomized clinical trials being a gold standard in many cases. 
However, due to recent advances in omics-technologies, medical 
imaging, comprehensive electronic health records, and smart devices,
medical research as well as clinical practice are quickly changing 
into big data-driven fields. As such, the healthcare domain as a 
whole - doctors, patients, management, insurance, and politics -
can significantly profit from current advances in Big Data 
technologies, and from analytics\cite{TR04}.
%%%%%%%%%%%%%%%%%%%%%%%%%%%%%%%%%%%%%%%%%%%%%%%%%%%%%%%%%%%%%%%%%%%
\subsection{Machine Learning}
Many healthcare applications would significantly benefit from the
processing and analysis of multimodal data - such as images, 
signals, video, 3D models, genomic sequences, reports, etc. 
Advanced machine learning systems can be used to learn and relate
information from multiple sources and identify hidden correlations 
not visible when considering only one source of data. For instance,
combining features from images (e.g. CT scans, radiographs) and text
(e.g. clinical reports) can significantly improve the performance 
of solutions\cite{TR04}.
%%%%%%%%%%%%%%%%%%%%%%%%%%%%%%%%%%%%%%%%%%%%%%%%%%%%%%%%%%%%%%%%%%%

\section{Challenges}
Large biomedical organizations typically save their discoveries confidential 
due to the costs associated in developing the drug throughout its life cycle 
almost 12 years it may take for a medication to be ready on prescription
pad from discovery and also very costly deal about \$4 billion to spent for
the whole process, because of costly investments, it is not feasible option 
to share the secrets of upcoming blockbuster drugs, on top of it only ten
percent of drugs finish its life cycle and come to market\cite{TR06}.

Although there is already a 
huge amount of healthcare data around the world and while it is 
growing at an exponential rate, nearly all the data is stored in
individually. Data collected by a clinic or by a hospital is mostly 
kept within the boundaries of the healthcare provider. Moreover, 
data stored within a hospital is hardly ever integrated across 
multiple IT systems. For example, if we consider all the available 
data at a hospital from a single patient's perspective, information 
about the patient will exist in the EMR system, laboratory, imaging 
system and prescription databases. Information describing which 
doctors and nurses attended to the specific patient will also exist.
However, in most of cases, every data source mentioned here is stored
in separate silos. Thus, deriving insights and therefore value from 
the aggregation of these data sets is not possible at this stage. It
is also important to realize that in today's world a patient's 
medical data does not only reside within the boundaries of a 
healthcare provider. The medical insurance and pharmaceuticals 
industries also hold information about specific claims and the
characteristics of prescribed drugs respectively. Increasingly,
patient-generated data from IoT devices such as fitness trackers, 
blood pressure monitors and weighing scales are also providing 
critical information about the day-to-day lifestyle
characteristics of an individual. Insights derived from such data
generated by the linking among EMR data, vital data, laboratory
data, medication information, symptoms (to mention some of these)
and their aggregation, even more with doctor notes, patient
discharge letters, patient diaries, medical publications, namely
linking structured with unstructured data, can be crucial to
design coaching programs that would help improve people's
lifestyles and eventually reduce incidences of chronic disease,
medication and hospitalization\cite{TR04}.
%%%%%%%%%%%%%%%%%%%%%%%%%%%%%%%%%%%%%%%%%%%%%%%%%%%%%%%%%%%%%%%%%%%
\section{conclusion}
The latest trends in big data creativities in health care is bringing 
confident Influence on clinical trials.  Increased relations between 
collected data elements and nomenclature should help in streamline of
trial designing and sharing of data. The process of standardization and
quality improvement work go side by side with a growing big data 
infrastructure applying guarantee benefits to information curation
for trials.

%%%%%%%%%%%%%%%%%%%%%%%%%%%%%%%%%%%%%%%%%%%%%%%%%%%%%%%%%%%%%%%%%%%
\begin{acks}

The authors would like to thank to Professor and TAs for guiding in
making the better paper.

\end{acks}

\bibliographystyle{ACM-Reference-Format}
\bibliography{report} 

\end{document}
